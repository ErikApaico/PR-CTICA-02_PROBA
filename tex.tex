\documentclass[12pt]{article}

\usepackage[utf8]{inputenc}

\usepackage{amsmath}

\usepackage{amsfonts}
\usepackage{amssymb}
\usepackage{makeidx}
\usepackage{graphicx}
\usepackage{wasysym}


\usepackage{amssymb}
\usepackage{amsmath}
\usepackage{graphics}
\usepackage{enumerate}

\begin{document}

\begin{titlepage}

\begin{center}

\vspace*{-1.5in}

{\fontsize{14}{15}\bf \selectfont UNIVERSIDAD NACIONAL DE SAN CRISTOBAL DE HUAMANGA}\\

\vspace*{0.5cm}

{\fontsize{12}{15}\bf \selectfont FACULTAD DE INGENIERÍA DE MINAS, GEOLOGÍA Y CIVIL\\ }

\vspace*{0.15in} ESCUELA DE FORMACIÓN PROFESIONAL CIENCIAS FÍSICO MATEMÁTICAS \\

\vspace*{0.2in}

\begin{figure}[htb]

\begin{center}

\includegraphics[width=4.5cm,height=6.5cm]{unsch.jpg}

\end{center}

\end{figure}

\begin{Large}

\textbf{CÁLCULO DE PROBABILIDAD }\\



\end{Large}

\vspace*{0.3in}

\begin{large}

RESOLUCIÓN DE EJERCICIOS\\

\rule{20mm}{0.1mm}

$\capricornus\pisces$ %Para que salga figuras de los signos Zodiacos utiliza este paquete %“\usepackage{wasysym} %para figuras especiales”

\rule{20mm}{0.1mm}

\end{large}

\vspace*{0.3in}

\vspace*{0.1in}

\end{center}

\begin{flushleft}
{\bf PROFESOR:} Jackson Macoy Romero Plasencia\\
\vspace*{0.2in}

{\bf ALUMNO:} Apaico Alvarado Erik Orlando\\


\end{flushleft}

\begin{center}

\begin{large}

\vspace*{0.50in}

Ayacucho-Peru\\

{\bf (2019)}

\end{large}





\title{Hoja de práctica 2\\
       \fntb{Resolución}
       }

\date{}
\begin{document}
\begin{enumerate}
\maketitle
\setcounter{enumi}{4}
\item  Dos personas A y B se distribuyen al azar en tres oficinas numerada 1, 2
y 3. Si las dos personas pueden estar en la misma oficina, defina un espació muestral adecuado.\\
{\bf Solución:} \\
$\Omega=\{(x,y)/x \in(1,2,3) &\bigwedge & y\in(A,B)\}$


\item Tres personas A , B y C se distribuyen al azar en dos oficinas numeradas
con 1 y 2. Describa un espacio muestral adecuado a este experimento, (a)
si los tres pueden estar en una misma oficina; (b) sí sólo se puede asignar una persona a cada oficina.\\
{\bf Solución:} \\
$\Omega=\{(x,y)/x \in(1,2) &\bigwedge & y\in(A,B,C)\}$\\
(a)    $\Omega_a=\{(A,B,C)=1 \bigwedge (A,B,C)=2\}$\\
(b)    $\Omega_b=\{A=1,B=1,C=1 \bigwedge A=2,B=2,C=2\}$\\

\item Durante el día, una máquina produce tres artículos cuya calidad individual,
definida como defectuoso o no defectuoso, se determina al final del
día. Describa el espacio muestral generado por la producción diaria.\\
{\bf Solución:} \\
$x_i, i=1,2,3$= árticulos , D= defectuosos, N=no defectuosos\\
$\Omega=\{(x_1,x_2,x_3)/x_i=(D,N) \}$ entondes\\
$\Omega_1=\{(x_1,D);(x_2,N);(x_3,N);
(x_1,N);(x_2,D);(x_3,N);\\
(x_1,N);(x_2,N);(x_3,D);
(x_1,D);(x_2,D);(x_3,N);
(x_1,D);(x_2,N);(x_3,D);\\
(x_1,N);(x_2,D);(x_3,D);
(x_1,D);(x_2,D);(x_3,D); \}$

\item El ala de un avión se ensambla con un número grande de remaches. Se inspecciona
una sola unidad y el factor de importancia es el número de remaches
ches defectuosos. Describa el espacio muestral.\\
{\bf Solución:} \\
 D=remaches defectuosos, N=remaches no defectuosos\\
$\Omega=\{x_i,i=1,2,..n/x_i=N \}$

\item Suponga que la demanda diaria de gasolina en una estación de servicio está acotada por 1,000 galones, que se lleva a un registro diario de venta.
Describa el espacio muestral. \\
{\bf Solución:} \\
 $x_i$=galones\\
$\Omega=\{0\leq x_i \leq1000\}$

\item Se desea medir la resistencia al corte de dos puntos de soldadura. Suponiendo
que el límite superior está dado por U, describa el espacio muestral.?\\
{\bf Solución:} \\
R=resistencia
$\Omega=\{R/ R\in[0,U]\}$

\item De un grupo de transistores producidos bajo condiciones similares, se escoge una sola unidad, se coloca bajo prueba en un ambiente similar a su uso diseñado y luego se prueba hasta que falla. Describir el espacio muestral.\\
{\bf Solución:} \\
t= iempo de vida del transitor\\
$\Omega=\{0\leqt t\leq T\}$

\item En el problema 11. (a) suponga que el experimento consiste en extraer dos
transistores y se prueba hasta que fallan. Describir el espacio muestral
(b) suponga que el experimento consiste en escoger 5 transistores y se prueba hasta que fallan. Describir el espacio muestral.\\
{\bf Solución:} \\
(a) $\Omega=\{(x,y)/0\leqt x,y\leq T\}$ \\
(b) $\Omega=\{(a,b,c,d,e)/0\leqt a,b,c,d,\leq T\}$

\item Una urna contiene cuatro fichas numeradas: 2,4,6, y 8 ; una segunda urna
contiene cinco fichas numeradas: 1,3,5,7, y 9. Sea un experimento aleato
rio que consiste en extraer una ficha de la primera urna y luego una ficha
de la segunda urna, describir el espacio muestral.\\
{\bf Solución:} \\
 $\Omega=\{(x,y)/x=2,4,6,8 &&\bigwedge y=1,3,5,7,9 \}$


\item Una urna contiene tres fichas numeradas: 1,2,3; un experimento consiste
en lanzar un dado y luego extraer una ficha de la urna. Describir el espació muestral.\\
{\bf Solución:} \\
 $\Omega=\{(x,y)/x=(1,2,3) &&\bigwedge y=1,2,3,4,5,6 \}$
 
\item Una línea de producción clasifica sus productos en defectuosos "D" y no
defectuosos "N". De un almacén donde guardan la producción diaria de esta
línea, se extraen artículos hasta observar tres defectuosos consecutivos o hasta que se hayan verificado cinco artículos. Construir el espacio muestral.\\
{\bf Solución:} \\
N= no defectuosos ,  D=defectuosos
 $\Omega=\{DDDDD,DDDND,DNDDD,NDDDD,DDDDN,NNDDDD,DDDNN,NDDDN \}$
 

\item Lanzar un dado hasta que ocurra el número 4. Hallar el espacio muestral
asociado a este experimento.\\
{\bf Solución:} \\
A=4 Y B\neq4 \\
 $\Omega=\{A,BA,BBA,BBBA,... \}$
  
\item Una moneda se lanza tres veces. Describa los siguientes eventos:\\
A :"ocurre por lo menos 2 caras".\\
B :"ocurre sello en el tercer lanzamiento".\\
C :"ocurre a lo más una cara".\\
{\bf Solución:} \\
C=cara   S=sello\\
 A$=\{CCC,CCS,SCC,CSC \}$\\
 B$=\{CCS,CSS,SSS,SCS \}$\\
 C$=\{CSS,SSS,SSC,SCS \}$\\
  
\item En cierto sector de Lima, hay cuatro supermercados (numeradas 1,2,3,4).
Seis damas que viven en ese sector seleccionan al azar y en forma independiente,
un supermercado para hacer sus compras sin salir de su sector
(a) Dar un espacio muestral adecuado para este experimento.
(b) Describir los siguientes eventos: \\
A : "Todas las damas escogen uno de los tres primeros supermercados"\\
B : "Dos escogen el supermercado N° 2 , dos el supermercado N°3 y las otras dos el N° 4".\\
C : "Dos escogen el supermercado N° 2 y las otras diferentes supermercados".\\
{\bf Solución:} \\
supermercados=(1,2,3,4)  damas=(a,b,c,d,e,f)\\
(a)  .$\Omega=\{(x,y)/x=(1,2,3,4) &&\bigwedge y=a,b,c,d,e,f\}$ \\
A$=\{(1,a);(1,b);(1,c);(1,d);(1,e);(1,f);(2,a);(2,b);(2,c);(2,d);\\(2,e);(2,f);(3,a);(3,b);(3,c);(3,d);(3,e);(3,f) \}$\\
 B$=\{(2,a);(2,b);(3,c);(3,d);(4,e);(4,f);(2,c);(2,d);(3,f);(3,e);(4,a);\\(4,b);(2,f);(2,e);(3,a)(3,e);(4,c);(4,d) \}$\\
 C$=\{(2,a);(2,b);(1,c);(3,d);(4,f); \}$\\
 
\item Tres máquinas idénticas que funcionan independientemente se mantienen -
funcionando hasta darle de baja y se anota el tiempo que duran. Suponer
que ninguno dura más de 10 años.\\
(a) Definir un espacio muestral adecuado para este experimento.\\
(b) Describir los siguientes eventos:\\
A : "Las tres máquinas duran más de 8 años".\\
B : "El menor tiempo de duración de los tres es de 7 años".\\
C : "Ninguna es dada de baja antes de los 9 años".\\
D : "El mayor tiempo de duración de los tres es de 9 años".\\
{\bf Solución:} \\
$x_i, i=1,2,3$ =máquinas\\
(a)  .$\Omega=\{(x_1,x_2,x_3)=(1,2,3N) \}$\\
(b)  \\
A$=\{x_1,x_2,x_3>8\}$\\
B$=\{x_1,x_2,x_3>7\}$\\
C$=\{9\leq x_1,x_2,x_3 \leq 10\}$\\
D$=\{x_1,x_2,x_3=9\}$\\

\item En el espacio muestral del problema 4, describe los siguientes eventos:\\
A : "Ocurre al menos 2 artículos no defectuosos".\\
B : "Ocurre exactamente 2 artículos no defectuosos".\\
{\bf Solución:} \\
$\ A=\lbrace DNN,NDN,NND,NNN\rbrace$\\
$\ B=\lbrace DNN,NDN,NND\rbrace$


\item En el problema 16, describir el evento, "se necesitan por lo menos 5 lan
zamientos".
{\bf Solución:} \\
Se necesitan por lo menos 5 lanzamientos = $\lbrace xxxx4.xxxxx4,xxxxxx4,....\rbrace$ ; donde x = obtener un número diferente de 4 .

\item El gerente general de una firma comercial, entrevista a 10 aspirantes a
un puesto. Cada uno de los aspirantes es calificado como: Deficiente, regular, Bueno, Excelente.\\
(a) Dar un espacio muestral adecuado para este experimento .\\
(b) Describir los siguientes eventos.\\
A : "Todos los aspirantes son calificados como deficientes o excelentes".\\
B : "Sólo la última persona extrevistada es calificado como excelente'.\\
{\bf Solución:} \\
D= deficiente, R=regular, B=bueno, E=exelente\\
(a)  .$\Omega=\{x_i, i=1,2,...,10/ x_i \in (D,R,B,E)\}$\\
(b)\\
.$\Omega_A=\{x_i, i=1,2,...,10/ x_i \in (D,E)\}$\\
.$\Omega_B=\{(x_i,E)/ x_i \in (D,R,B) i=1,2,...,9\}$\\

\item Considere el experimento de contar el número de carros que pasan por un
punto de una autopista. Describa los siguientes eventos:\\
A : "Pasan un número par de carros".\\
B : "El número de carros que pasan es múltiplo de 6 ".\\
C : "Pasan por lo menos 20 carros".\\
D : "Pasan a lo más 15 carros".\\
{\bf Solución:} \\
$\omega_A=\lbrace0,2,4,6,8,10,....\rbrace$\\
$\omega_B=\lbrace0,6,12,18,....\rbrace$\\
$\omega_C=\lbrace20,21,22,23,24,...\rbrace$\\
$\omega_D=\lbrace1,2,3,4,.....,14,15\rbrace$\\


\item En el problema 12. Describir los siguientes eventos. (1) en la parte (a).\\
A : "Los dos transistores duran a lo más 2,000 horas".\\
B : "El primero dura más de 2,000 horas, el otro menos de 3,000 horas".
(2) En la parte (b).\\
A : "Los cinco duran por lo menos 1,000 horas pero menos de 2,000 horas".\\
B : "El primero dura más de 2,000 horas, los demás a lo más 2,500 horas".\\
{\bf Solución:} \\
$\ A=\lbrace (x,y) \diagup 0\leq x,y\leq2000\rbrace$ , donde x: el tiempo de falla del transistor designado como número 1; y: el tiempo de falla del transistor designado como número 2.\\


$\ B=\lbrace (x,y) \diagup 2000\leq x<\infty ; 0\leq y\leq3000\rbrace$ \\




$\ C=\lbrace ( x_{1},x_{2},x_{3},x_{4},x_{5}) \diagup 1000\leqx_{1},x_{2},x_{3},x_{4},x_{5}<2000\rbrace$\\
$\ D=\lbrace ( x_{1},x_{2},x_{3},x_{4},x_{5}) \diagup 2000\leq x_{1}<\infty ; 0\leq x_{2},x_{3},x_{4},x_{5}\leq2500\rbrace$\\


\end{enumerate}
\end{document}